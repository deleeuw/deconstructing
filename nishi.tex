% Options for packages loaded elsewhere
\PassOptionsToPackage{unicode}{hyperref}
\PassOptionsToPackage{hyphens}{url}
%
\documentclass[
  12pt,
]{article}
\usepackage{amsmath,amssymb}
\usepackage{iftex}
\ifPDFTeX
  \usepackage[T1]{fontenc}
  \usepackage[utf8]{inputenc}
  \usepackage{textcomp} % provide euro and other symbols
\else % if luatex or xetex
  \usepackage{unicode-math} % this also loads fontspec
  \defaultfontfeatures{Scale=MatchLowercase}
  \defaultfontfeatures[\rmfamily]{Ligatures=TeX,Scale=1}
\fi
\usepackage{lmodern}
\ifPDFTeX\else
  % xetex/luatex font selection
\fi
% Use upquote if available, for straight quotes in verbatim environments
\IfFileExists{upquote.sty}{\usepackage{upquote}}{}
\IfFileExists{microtype.sty}{% use microtype if available
  \usepackage[]{microtype}
  \UseMicrotypeSet[protrusion]{basicmath} % disable protrusion for tt fonts
}{}
\makeatletter
\@ifundefined{KOMAClassName}{% if non-KOMA class
  \IfFileExists{parskip.sty}{%
    \usepackage{parskip}
  }{% else
    \setlength{\parindent}{0pt}
    \setlength{\parskip}{6pt plus 2pt minus 1pt}}
}{% if KOMA class
  \KOMAoptions{parskip=half}}
\makeatother
\usepackage{xcolor}
\usepackage[margin=1in]{geometry}
\usepackage{longtable,booktabs,array}
\usepackage{calc} % for calculating minipage widths
% Correct order of tables after \paragraph or \subparagraph
\usepackage{etoolbox}
\makeatletter
\patchcmd\longtable{\par}{\if@noskipsec\mbox{}\fi\par}{}{}
\makeatother
% Allow footnotes in longtable head/foot
\IfFileExists{footnotehyper.sty}{\usepackage{footnotehyper}}{\usepackage{footnote}}
\makesavenoteenv{longtable}
\usepackage{graphicx}
\makeatletter
\def\maxwidth{\ifdim\Gin@nat@width>\linewidth\linewidth\else\Gin@nat@width\fi}
\def\maxheight{\ifdim\Gin@nat@height>\textheight\textheight\else\Gin@nat@height\fi}
\makeatother
% Scale images if necessary, so that they will not overflow the page
% margins by default, and it is still possible to overwrite the defaults
% using explicit options in \includegraphics[width, height, ...]{}
\setkeys{Gin}{width=\maxwidth,height=\maxheight,keepaspectratio}
% Set default figure placement to htbp
\makeatletter
\def\fps@figure{htbp}
\makeatother
\setlength{\emergencystretch}{3em} % prevent overfull lines
\providecommand{\tightlist}{%
  \setlength{\itemsep}{0pt}\setlength{\parskip}{0pt}}
\setcounter{secnumdepth}{5}
% definitions for citeproc citations
\NewDocumentCommand\citeproctext{}{}
\NewDocumentCommand\citeproc{mm}{%
  \begingroup\def\citeproctext{#2}\cite{#1}\endgroup}
\makeatletter
 % allow citations to break across lines
 \let\@cite@ofmt\@firstofone
 % avoid brackets around text for \cite:
 \def\@biblabel#1{}
 \def\@cite#1#2{{#1\if@tempswa , #2\fi}}
\makeatother
\newlength{\cslhangindent}
\setlength{\cslhangindent}{1.5em}
\newlength{\csllabelwidth}
\setlength{\csllabelwidth}{3em}
\newenvironment{CSLReferences}[2] % #1 hanging-indent, #2 entry-spacing
 {\begin{list}{}{%
  \setlength{\itemindent}{0pt}
  \setlength{\leftmargin}{0pt}
  \setlength{\parsep}{0pt}
  % turn on hanging indent if param 1 is 1
  \ifodd #1
   \setlength{\leftmargin}{\cslhangindent}
   \setlength{\itemindent}{-1\cslhangindent}
  \fi
  % set entry spacing
  \setlength{\itemsep}{#2\baselineskip}}}
 {\end{list}}
\usepackage{calc}
\newcommand{\CSLBlock}[1]{\hfill\break#1\hfill\break}
\newcommand{\CSLLeftMargin}[1]{\parbox[t]{\csllabelwidth}{\strut#1\strut}}
\newcommand{\CSLRightInline}[1]{\parbox[t]{\linewidth - \csllabelwidth}{\strut#1\strut}}
\newcommand{\CSLIndent}[1]{\hspace{\cslhangindent}#1}
\ifLuaTeX
  \usepackage{selnolig}  % disable illegal ligatures
\fi
\IfFileExists{bookmark.sty}{\usepackage{bookmark}}{\usepackage{hyperref}}
\IfFileExists{xurl.sty}{\usepackage{xurl}}{} % add URL line breaks if available
\urlstyle{same}
\hypersetup{
  pdftitle={Deconstructing Multiple Correspondence Analysis},
  pdfauthor={Jan de Leeuw},
  hidelinks,
  pdfcreator={LaTeX via pandoc}}

\title{Deconstructing Multiple Correspondence Analysis}
\author{Jan de Leeuw}
\date{First created on March 28, 2022. Last update on October 09, 2023}

\begin{document}
\maketitle
\begin{abstract}
This chapter has two parts. In the first part we review the history of Multiple Correspondence Analysis (MCA) and Reciprocal Averaging Analysis (RAA). Specifically we comment on the 1950 exchange between Burt and Guttman about MCA, and the distinction between scale analysis and factor analysis. In the second part of the chapter we construct an MCA alternative, called Deconstructed Multiple Correspondence Analysis (DMCA), which is useful in the discussion of ``dimensionality'', ``variance explained'', and the ``Guttman effect'', concepts that were important in the history covered in the first part.
\end{abstract}

\section{Notation}\label{notation}

Let us start by defining some of the notation used in this paper. We have \(i=1,\cdots n\) observations on each of \(j=1,\cdots,m\) categorical variables, where variable \(j\) has \(k_j\) categories. We use \(k_\star\) for the sum of the \(k_j\), while the maximum number of categories over all variables is \(k_+=\max(k_1,\cdots,k_m)\). We also define \(m_s\), with \(s=1,\cdots,k_+\), where \(m_s\) is the number of variables with \(k_j\geq s\). Thus both \(m_1\) and \(m_2\) are always equal to \(m\). Also \(\sum_{s=1}^{k_+} m_s=k_\star\). The fact that variables can have a different number of categories is a major notational nuisance. If they
all have the same number of categories \(k\) then \(k_+=k\), \(k_\star=mk\),
and all \(m_s\) are equal to \(m\).

The data are coded as \(m\) indicator matrices \(G_j\), with \(\{G_j\}_{ik}=1\) if and only if object \(i\) is in category \(k\) of variable \(j\) and \(\{G_j\}_{ik}=0\) otherwise. The \(G_j\) are \(n\times k_j\), zero-one, and columnwise orthogonal (because the categories are mutually exclusive). If we concatenate the \(G_j\) horizontally we have the \(n\times k_\star\) matrix \(G\), which we also call the indicator matrix (in French data analysis it is the ``tableau disjonctif complet'', in Nishisato (1980) it is the ``response-pattern table''). The Burt table (``tableau de Burt''), is the \(k_\star\times k_\star\) cross product matrix \(C=G'G\). The univariate marginals are in the diagonal matrix \(D=\text{diag}(C)\). The normalised Burt table is the matrix \(E=m^{\frac12}D^{-\frac12}CD^{-\frac12}\).

Although we introduced \(G, C, D\) and \(E\) as partitioned matrices of real
numbers, it is also useful to think of them as matrices with matrices
as elements. Thus \(C\), for example, is an \(m\times m\) matrix with as elements
the matrices \(C_{j\ell}^{\ }=n^{-1}G_j'G_\ell^{\ }\), and \(G\) is an \(1\times m\)
matrix with as its \(m\) elements \(G_j\). Note that because we have divided
the cross product by \(n\), all \(C_{j\ell}\), and thus all \(D_j=C_{jj}\), add up to one.

In the paper we often use the \emph{direct sum} of matrices. If \(A\) and \(B\) are matrices, then their direct sum is
\begin{equation}
A\bigoplus B=\begin{bmatrix}A&0\\0&B\end{bmatrix},
\label{eq:oplus}
\end{equation}
and if \(A_r\) are \(s\) matrices, then \(\bigoplus_{r=1}^s A_r\) is block-diagonal
with the \(A_r\) as diagonal submatrices.

\section{Introduction}\label{introduction}

Multiple Correspondence Analysis (MCA) can be introduced in many different ways.

\emph{Mathematically} MCA is the Singular Value Decomposition (SVD) of \(m^{-\frac12}Gy=\sqrt{\lambda}x\) and \(m^{-\frac12}G'x=\sqrt{\lambda}Dy\), the eigenvalue problem \(Ey=\lambda^2 y\) for the normalised Burt table, and the Eigenvalue Decomposition (EVD) of \(m^{-1}G'D^{-1}Gx=\lambda^2 x\) for the average projector. Using \(m\) in the equations seems superfluous, but it guarantees that \(0\leq\lambda\leq 1\).

\emph{Statistically} MCA is a scoring method that minimises the within-individual and maximises the between-individuals variance, it is a graphical biplot method that minimises the distances between individuals and the categories of the variables they score in, it is an optimal scaling method that maximises the largest eigenvalue of the correlation matrix of the transformed variables, and that linearises the average regression of one variable with all the others. It can also be presented as a special case of Homogeneity Analysis, Correspondence Analysis, and Generalised Canonical Correlation Analysis. See, for example, the review article by Tenenhaus and Young (1985).

It is of some interest to trace the origins of these various MCA formulations, and to relate them to an interesting exchange in the 1950's between two of the giants of psychometrics on whose proverbial shoulders we still stand. In 1950 Sir Cyril Burt published, in his very own British Journal of Statistical Psychology, a great article introducing MCA as a form of factor analysis of qualitative data (Burt (1950)). There are no references in the paper to earlier occurances of MCA in the literature. This prompted Louis Guttman to point out in a subsequent issue of the same journal that the relevant equations were already presented in great detail in Guttman (1941). Guttman assumed Burt had not seen the monograph (Horst (1941)) in which his chapter was published, because of the communication problems during the war, which caused ``only a handful of copies to reach Europe'' (Guttman (1953)). Although the equations and computations given by both Burt and Guttman were identical, Guttman pointed out differences in interpretation between their two approaches. These differences will especially interest us in the present paper. They were also discussed in Burt's reaction to Guttman's note (Burt (1953)). The three papers are still very readable and instructive, and in the first part of the present paper we'll put them in an historical context.

\section{History}\label{history}

\subsection{Prehistory}\label{prehistory}

The history of MCA has been reviewed in De Leeuw (1973), Benz
